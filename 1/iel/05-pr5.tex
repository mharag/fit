\section{Příklad 5}
% Jako parametr zadejte skupinu (A-H)
\patyZadani{G}

\subsection{Zostrojenie diferencialnej rovnice 1. radu}

Vieme ze pre RL obvod platia nasledujuce vztahy.

\begin{align}
    i_L &= i_R = \frac{u_R}{R} \\
    u_R + u_L - U &= 0 (2)\\
    i_L' &= \frac{u_L}{L}, i_L(t) = 0
\end{align}

Ked $u_R$ zo vztahu (1) dosadime do vztahu (2) a potom $u_L$ zo vztahu (2) dosadime do vztahu (3), 
dostaneme takuto diferencialnu rovnicu

\begin{align}
    i_L' = \frac{U - i_L R}{L}  \iff\\
    i_L' + i_L \frac{R}{L} = \frac{U}{L}
\end{align}

Rovnica (6) je nehomogenna diferencialna rovnica 1. radu. Mozeme si pomoct ocakavanym riesenim.

\begin{align}
    i_L(t) = k(t) \cdot e^{\lambda t}
\end{align}

Substituujeme $i_L' -> \lambda$ a $i_L -> 1$

\begin{align}
    \lambda  + \frac{R}{L} &= 0 \iff\\
    \lambda &= -\frac{R}{L}
\end{align}

Ak teraz dosadime $\lambda$ do ocakavaneho riesenia dostaneme 

\begin{align}
    i_L(t) = k(t) \cdot e^{-\frac{R}{L}t}
\end{align}

Rovnicu (9) zderivujeme a $i_L$ spolu s $i_L'$ dosadime do (4)

\begin{align}
    i_L'(t) &= k'(t) \cdot e^{-\frac{R}{L}t} + k(t) \cdot \frac{-R}{L} \cdot e^{-\frac{R}{L}t}\\
    k'(t) e^{-\frac{R}{L}t} + k(t) \frac{-R}{L} e^{-\frac{R}{L}t} + \frac{R}{L} k(t) e^{-\frac{R}{L}t} &= \frac{U}{L} \iff\\
    k'(t) e^{-\frac{R}{L}t} &= \frac{U}{L} \iff\\
    k'(t) &= \frac{U}{L} e^{\frac{R}{L}t}
\end{align}

Poslednu rovnicu integrujeme a dostaneme

\begin{align}
    k(t) &= \frac{U}{R} e^{\frac{R}{L}t} + k
\end{align}

(14) dosadime do (9) a mame hotove analiticke riesenie.

\begin{align}
    i_L(t) = (\frac{U}{R} e^{\frac{R}{L}t} + k) \cdot e^{-\frac{R}{L}t} = \frac{U}{L} + k e^{-\frac{R}{L}t}
\end{align}

Dosadime nase parametre

\begin{align}
    i_L(t) = \frac{\SI{20}{\volt}}{\SI{25}{\ohm}} + k e^{-\frac{\SI{25}{\ohm}}{\SI{50}{\henry}}t} = \frac{4}{5} + ke^{-2t}(A)
\end{align}

Po dosadeni pocatocnej podmienky dostaneme aj k

\begin{align}
    i_L(0) = \SI{8}{\ampere} = \frac{4}{5} + k(A) \impies \\
    k = \frac{36}{5}
\end{align}

Ostava nam uz vykonat skusku

\begin{align}
    i_L(t) &= \frac{4}{5} + \frac{36}{5}e^{-2t} (A) \implies\\
    i_L'(t) &= -\frac{18}{5}e^{-2t}
    \\
    -\frac{18}{5}e^{-2t} +  \frac{1}{2}(\frac{4}{5} + \frac{36}{5}e^{-2t})  &= \frac{2}{5} \iff\\
    \frac{2}{5} &= \frac{2}{5}
\end{align}